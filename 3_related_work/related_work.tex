
\chapter{Related work}
  
  There have been several attempts to provide efficient memory management for virtual machines. The most prominent used approach is of 
that Ballooning \cite{waldspurger2002memory}. Ballooning is an mechanism that reclaims pages considered least valuable by the guest OS 
running inside the virtual machine. This allows the VM to decide which memory pages to release instead of the host trying to determine this.
Gerḿan Molt́o \cite{molto2013elastic} expands the idea of Ballooning to provide a system to monitor the VM memory and apply vertical 
elasticity rules in order to dynamically change its memory size by using the memory ballooning technique provided the KVM hypervisor. 
Overdriver \cite{williams2011overdriver} on the other hand presents a system that handles all durations of memory overload. It adapts its 
mitigation strategy to balance the trade offs between migration and cooperative swap to handle memory overcommitments. Ex-Tmem 
\cite{venkatesan2014ex} stores clean pages in a two-level buffering hierarchy with locality-aware data placement and replacement. It 
enables memory-to-memory swapping by using non-volatile memory and eliminates expensive I/O caused by swapping.
  
  Looking at researches that have looked at resource provisioning, CloudScale \cite{shen2011cloudscale} can resolve scaling conflicts 
between applications using migration, and integrates dynamic CPU voltage/frequency scaling to achieve energy savings with minimal effect on 
application SLOs. Tim Dornemanna\cite{dornemann2009demand} proposed a solution that automatically schedules workflow steps to underutilized 
hosts and provides new hosts using cloud computing infrastructures in peak-load. The system was based on BPEL to support on-demand resource 
provisioning. Aneka \cite{calheiros2012aneka}, is a platform for developing scalable applications on the Cloud, that supports provisioning 
resources from different sources and supporting different application models. It support the integration between Desktop Grids and Clouds. 
Elastic Application Container (EAC) \cite{he2012elastic} is a virtual resource unit for delivering better resource efficiency and more 
scalable cloud applications. 

  The idea behind derivative cloud for our discussions have emerged from SpotCheck \cite{sharma2015spotcheck}  provides the illusion of an 
IaaS platform that offers always-available VMs on demand for a cost near that of spot servers using a nest setup of virtual execution 
environments. 

  Although there are several existing works in elastic resource provisioning (including memory) for virtual machines as listed above. There 
has been no attempt to provide an deterministic memory management policy for containers. To our knowledge this is our first attempt to do 
so. 
