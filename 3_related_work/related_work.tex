
\chapter{Related work}
  The following is a comprehensive list of related works. Although most of these related works are motivating factors to that 
  drive our work, none of them are similar works.
  
  \section{Memory management in virtualized environments}
    There have been several attempts to provide efficient memory management for virtual machines. The most prominent used approach is of 
    that Ballooning \cite{waldspurger2002memory}. Ballooning is an mechanism that reclaims pages considered least valuable by the guest OS 
    running inside the virtual machine. This allows the VM to decide which memory pages to release instead of the host trying to determine this.
    Gerḿan Molt́o \cite{molto2013elastic} expands the idea of Ballooning to provide a system to monitor the VM memory and apply vertical 
    elasticity rules in order to dynamically change its memory size by using the memory ballooning technique provided the KVM hypervisor. 
    Overdriver \cite{williams2011overdriver} on the other hand presents a system that handles all durations of memory overload. It adapts its 
    mitigation strategy to balance the trade offs between migration and cooperative swap to handle memory overcommitments. Ex-Tmem 
    \cite{venkatesan2014ex} stores clean pages in a two-level buffering hierarchy with locality-aware data placement and replacement. It 
    enables memory-to-memory swapping by using non-volatile memory and eliminates expensive I/O caused by swapping.
  
  \section{Resource provisioning in virtualized environments}
    Looking at researches that have looked at resource provisioning, CloudScale \cite{shen2011cloudscale} can resolve scaling conflicts 
    between applications using migration, and integrates dynamic CPU voltage/frequency scaling to achieve energy savings with minimal effect on 
    application SLOs. Tim Dornemanna\cite{dornemann2009demand} proposed a solution that automatically schedules workflow steps to underutilized 
    hosts and provides new hosts using cloud computing infrastructures in peak-load. The system was based on BPEL to support on-demand resource 
    provisioning. Aneka \cite{calheiros2012aneka}, is a platform for developing scalable applications on the Cloud, that supports provisioning 
    resources from different sources and supporting different application models. It support the integration between Desktop Grids and Clouds. 
    Elastic Application Container (EAC) \cite{he2012elastic} is a virtual resource unit for delivering better resource efficiency and more 
    scalable cloud applications. 

  \section{Nested virtualization}
    A derivative cloud is a nested setup, virtual machines nested in virtual machines\cite{williams2012xen} or containers deployed in virtual machine 
    \cite{sharma2015spotcheck, gcp}, the latter being the focus of this work.  
    
  \section{Hypervisor managed caches}
    There has been extensive literature centered around hypervisor caching
    \cite{lu2007virtual, mishra2014comparative, schopp2006resizing}.
    There are several types of hypervisor caching addressed in various works, however the framework 
    we have used to build upon in our implementation is described below. 
    
    \subsection{Transcendent Memory}
      Transcendent Memory (Tmem)\cite{magenheimer2009transcendent} is an approach make proper utilization of underutilized
      memory present in Linux based system by provisioning an in-memory cache. This was then extended to virtualized setups
      to be able to build caches for individual VMs. However there have been attempts \cite{venkatesan2014ex} to make this 
      single level Tmem cache to an hybrid multi-level setup in a non-virtualized by environments. Tmem caches however only
      cache clean disk pages. They don't cache dirty disk or anonymous pages.
      
   \subsection{Hypervisor cache partitioning}
    
       On top of this there has been work carried out to partition hypervisor caches based on 
       application SLAs\cite{schopp2006resizing, koller2015centaur}. In these works, each application 
       is treated as a VM and provisioned accordingly. Centuar\cite{koller2015centaur} proposes a single
       level cache partitioned on a per VM bases by satisfying individual application objects or a global
       hypervisor level policy. SDC\cite{schopp2006resizing} follows a similar approach but provides 
       heuristics for even a multi-level cache and how application requirements could be satisfied. Both
       There approaches make use of MRC construction techniques\cite{zhou2004dynamic, zhao2011low, waldspurger2015efficient}.  
  

  \section{Conclusions}
    Although there are several existing works in elastic resource provisioning (including memory) for virtual machines as listed above. There 
    has been no attempt to provide an deterministic memory management policy for containers. To our knowledge this is our first attempt to do 
    so.
    
    Applicability of hypervisor caching in a derivative cloud setup is hindered due to inability of existing frameworks to support this sort of
    provisioning, and also we would like to look at the overall picture of memory management at all levels of the hierarchy along with the cache
    partitioning to satisfy application SLA. 

%   \subsection{Hypervisor managed caching}
%       \subsubsection{T-MEM cache}
%     
%    \subsection{Multilevel caches}
%     
%    \subsection{Application specific cache partitoning}
%       \subsubsection{MRC construction}
%       
%    