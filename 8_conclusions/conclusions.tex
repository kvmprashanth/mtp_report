
\chapter{Conclusions}
  
    We have made an initial attempt to understand memory management in Linux containers. We started off with purposing hypotheses based on 
theoretical evidences. We performed empirical analysis to verify the correctness of our purposed hypotheses. We also performed a few more 
empirical analysis to establish parts of memory management for which hypotheses couldn't be drawn. We then tried to extrapolate its 
implications in the real world applications running inside a derivative cloud environment. These implications strongly suggested that 
existing memory management techniques may impact higher provisioned containers negatively, when the system is under memory pressure. We 
conclude by purposing the requirements of a new desired policy that provides this notion of a differentiated reclamation to enforce 
deterministic allocation when the system is under memory pressure.  
  
 