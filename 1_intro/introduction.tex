\chapter{Introduction}
  
  Mention about derivative clouds in intro itself
  
  \section{Memory management in clouds}
     
    \subsection{Issues in native container environment}	
    \subsection{Amplification of issues in derivative cloud environment}
  
  \section{Caching in the cloud}
  
   \subsection{Drawbacks of caching in native (VM) cloud setups}
  
   \subsection{Hypervisor managed caching}
      
    \subsection{Issues of caching frameworks in derivative clouds}
      \subsubsection{Lack of framework support in derivative clouds}
      \subsubsection{Dual layers of isolated control}	
	 \paragraph{Derivative provider has no control over cache partioning}
	    Cache-level sentivity
	 \paragraph{Native provider has no control over application memory allocations}
	    Annoymous memory sensitivity
      \subsection{Application cache sensitivity is unaccounted}
    
  \section{Problem description}
  
   \subsection{Phase-1}
   In our work we wish to,
    
    \begin{enumerate}
      \item Understand the existing memory management policies used by Linux to manage containers.
      \item Purpose hypotheses on existing memory management in containers and verify them using experimental results.
      \item Extrapolate this understanding to analyze, how it impacts derivative cloud environment.
      \item Identify issues with existing policies and purpose requirements for a new policy
    \end{enumerate}

   \subsection{Phase-2}
   
   sadsa
  
  