
\chapter{Empirical Analysis in derived cloud}

  The list of questions mentioned in section:\ref{section_questions} are questions of interest that would help us understand the existing 
memory management techniques better. We have tried answer the listed questions by mapping them into appropriate experiments. Inferences 
were drawn based on the observations in the experiments. There were 3 different categories of questions we have tried to answer using 
empirical analysis in a native container environment as listed below,
  
  \begin{enumerate}
    \item Verify the correctness of our hypotheses
    \item Understand parts of memory management for which hypothesis couldn't be drawn
    %\item Understand the implications of existing memory management on application performance
  \end{enumerate}
  
  The configuration in Table:\ref{table_native_base} is the base configuration for all experiments in this section. Any changes the base 
configuration has been mentioned in the procedure of each of the experiment.

    \begin{table}	 
      \begin{center}
	\begin{tabular}{ l | c | c }
	  & Container-1 (M1) & Container-2 (M2) \\ 
	  \hline
	  \hline
	  Size of VM & \multicolumn{2}{c}{2 GB} \\	      
	  \hline
	  Workload & Memory Hogger & Memory Hogger \\
	  \hline
	  Hard Limit & 1000 MB & 1000 MB \\  
	  \hline
	  Soft Limit & 150 MB & 150 MB \\  
	  \hline
	  Memory Usage & 500 MB & 500 MB \\
	  \hline
	  Exceed & 350 MB & 350 MB \\
	  \hline 
	  External Pressure & \multicolumn{2}{l}{ 200 - 400 - 600 - 800 - 1000 MB} \\
	\end{tabular}	    
	\caption{Base Configuration}
	\label{table_native_base}
      \end{center}
    \end{table}
    
    Most experiments involved setting up of 2 containers. Workloads were used to introduce system memory pressure from containers. At 
this point there was no memory pressure in the system (free memory was still available). Now the external pressure using Stress was 
introduced after about 20s which created memory pressure in the system that triggered reclamation. The external pressure kept on increasing 
by 200 MB in intervals of 40s. Each interval had a gap of 10s for memory to be reassigned to containers.
  
  %For the sake of simplicity, questions of category 1 and 2 were answered using the native testbed using synthetic workloads as described 
%in section:\ref{section_testbed_native}. Questions of category 3 were answered using the derivative testbed using real workloads as 
%described in section:\ref{section_testbed_derivative}

  \section{Verification of hypotheses}
    
    
	
%%%%%%%%%%%%%%%%%%%%%%%%%%%%%%%%%%%%%%%%%%%%%%%%%%%%%%%%%%%%%%%%%%%%%%%%%%%%%%%%%%%%%%%%%%%%%%%%%%%%%%%%%%%%%%%%%%%%%%%%%%%%%%%%%%%%%%%%%%%
  
  \section{Conclusions}
    
    